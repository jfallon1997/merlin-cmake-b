\hypertarget{index_intro_sec}{}\section{Introduction}\label{index_intro_sec}
This is the introduction.\hypertarget{index_install_sec}{}\section{Installation}\label{index_install_sec}
\hypertarget{index_Linux}{}\subsection{Linux}\label{index_Linux}
The current method is to use cmake to build makefiles for Merlin 1\+: make a \char`\"{}build\char`\"{} folder outside the folder where Merlin was extracted. 2\+: run \char`\"{}cmake /path/to/merlin/folder\char`\"{} 3\+: run \char`\"{}ccmake /path/to/buildfolder\char`\"{} 4\+: Pick the options you wish to use. 5\+: run make -\/j $<$ ncpu $>$ 6\+: This will make the merlin library\hypertarget{index_Windows}{}\subsection{Windows}\label{index_Windows}
Needs to be updated. \hypertarget{index_OSX}{}\subsection{O\+SX}\label{index_OSX}
The optimal way to use Merlin on O\+SX is to get cmake to generate build files for Xcode. This of course requires that xcode is installed. By default cmake will generate unix makefiles. To generate Xcode files the appropriate generator must be specified when using cmake as\+:

cmake -\/G Xcode

One can then run\+:

xcodebuild

which will compile merlin. Other options exist such as \char`\"{}xcodebuild clean\char`\"{}, which will clear out built files.

\begin{DoxyAuthor}{Author}
Nick Walker 

Dirk Kruecker 

Andy Wolski 

Roger Barlow 

Adina Toader 

James Molson 

Haroon Rafique 

Sam Tygier 
\end{DoxyAuthor}
\begin{DoxyVersion}{Version}
4.\+9 
\end{DoxyVersion}
\begin{DoxyDate}{Date}
2001-\/2017 
\end{DoxyDate}
